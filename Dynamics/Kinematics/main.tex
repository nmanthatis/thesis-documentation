\section{Κινηματική}
\noindentΗ κίνηση του τρικοπτέρου ορίζεται πλήρως από το διάνυσμα θέσης του 
κέντρου μάζας του και από τον προσανατολισμό του ως προς κάποιο σύστημα 
αναφοράς. Στην παρούσα διπλωματική εργασία, η παραμέτρηση του προσανατολισμού 
έγινε με γωνίες \tl{Euler}.

\textbf{Αδρανειακό Σύστημα Αναφοράς Εδάφους} \(E : OXYZ\) \qquad Το σύστημα 
αναφοράς \(E\) βρίσκεται σταθερά δεμένο στην επιφάνεια της Γης και θεωρείται 
αδρανειακό. Το επίπεδο \(XY\) συμπίπτει με το έδαφος και ο άξονας \(Z\) έχει την
ίδια κατεύθυνση με το διάνυσμα του βάρους.

\textbf{Κεντροβαρικό Σύστημα Αναφοράς} \(B : Gxyz\) \qquad Το σύστημα αναφοράς 
\(B\) είναι σταθερά δεμένο στο κέντρο μάζας \(G\) του στερεού σώματος,  
μεταφέρεται και περιστρέφεται μαζί με αυτό. Οι διευθύνσεις των διανυσμάτων βάσης
του συστήματος αυτού, συμπίπτουν με τους άξονες ευαισθησίας των μετρητικών 
οργάνων. Επομένως, κάθε μέτρηση είναι εκφρασμένη σε αυτήν την βάση.

\subsection{Γωνίες \tl{Euler}}

\noindentΣε αντίθεση με την γραμμική μετατόπιση, η συνολική περιστροφική 
μετατόπιση ενός στερεού σώματος δεν μπορεί να προέλθει με απλή ολοκλήρωση του 
διανύσματος γωνιακής ταχύτητας ως προς ένα σύστημα αναφοράς \cite{natsiavas}. 
Για τον προσδιορισμό λοιπόν του προσανατολισμού ενός στερεού σώματος γίνεται 
χρήση των γωνιών \tl{Euler}. 

Για τον ορισμό των γωνιών \tl{Euler}, θεωρείται το σύστημα αναφοράς \(f(Oxyz)\), 
το οποίο είναι σταθερά δεμένο στο στερεό σώμα και περιστρέφεται μαζί με αυτό
ως προς την αρχή των αξόνων \(O\), και το σύστημα \(F(OXYZ)\), το οποίο 
παραμένει ακίνητο. Ο προσανατολισμός του συστήματος αναφοράς \(f\) (και άρα του 
στερεού σώματος) ως προς το σύστημα \(F\) μπορεί να επιτευχθεί με τρεις 
διαδοχικές γωνιακές μετατοπίσεις, με περιορισμό ότι δύο διαδοχικές περιστροφές 
δεν μπορούν να πραγματοποιηθούν γύρω από ίδιο ή παράλληλο άξονα. Η αλληλουχία 
των περιστροφών, που επιλέχθηκε, φαίνεται στα σχήματα (\ref{fig:rot_z}),
(\ref{fig:rot_y}), (\ref{fig:rot_x}). 
\begin{itemize}
    \item Περιστροφή του συστήματος αναφοράς \(f\) κατά γωνία εκτροπής \(\psi\) 
    (\tl{yaw angle}) γύρω από το σταθερό άξονα \(OZ (F \to f^\prime) \). Με μπλέ
    χρώμα απεικονίζεται το επίπεδο περιστροφής.

    \begin{figure}[h]%\kern-1.5ex
        \tikzset{cross/.style={
            cross out, draw=black, minimum size=2*(#1-\pgflinewidth),
            inner sep=0pt, outer sep=0pt}, cross/.default={1pt}}
        \centering
        \tdplotsetmaincoords{70}{170}
        \begin{tikzpicture}[scale = 2.5, tdplot_main_coords]
            % first plane
            \draw[thin] (1, 0, 0) -- (1, 1, 0) -- (-1, 1, 0) -- (-1, 0, 0) -- 
                cycle;
            \fill[blue!35, opacity=0.6] (1, 0, 0) -- (1, 1, 0) -- (-1, 1, 0) -- 
                (-1, 0, 0);
            
            \draw (0, 0, 0) coordinate (O) node[above right]{\(O\)};
            \draw[thick, -stealth] (0, 0, 0) -- (1, 0, 0) coordinate (a) 
                node[left] {\(X\)};
            \draw[thick, -stealth] (0, 0, 0) -- (0, 1, 0) coordinate (b)
                node[below] {\(Y\)}; 
            \draw[thick, -stealth] (0, 0, 0) -- (0, 0, 1) coordinate (c) 
                node[above] {\(Z, z^{\prime}\)};
            
            \draw[thick, -stealth] (0, 0, 0) -- (0.866, 0.5, 0) coordinate 
                (d) node[below] {\(x^{\prime}\)};
            \draw[thick, -stealth] (0, 0, 0) -- (-0.5, 0.866, 0) coordinate 
                (e) node[above right] {\(y^{\prime}\)};
    
            \draw[thick, draw=red, -stealth] (0, 0, 0) -- (0, 0, 0.9) 
                node[below right, red] {\(\dot{\psi}\)};

            \draw pic[thick, ->, "$\psi$", draw=red, text=red, angle radius=40, 
                angle eccentricity=1.4] {angle = a--O--d};
            \draw pic[thick, ->, "$\psi$", draw=red, text=red, angle radius=20,
                angle eccentricity=1.4] {angle = b--O--e};
        \end{tikzpicture}
        \caption{Περιστροφή Συστήματος προς άξονα \(OZ\).}
        \label{fig:rot_z}
    \end{figure}

    \item Περιστροφή του συστήματος αναφοράς \(f^\prime\) κατά γωνία πρόνευσης 
    \(\theta\) (\tl{pitch angle}) γύρω από τον τρέχων άξονα \(Oy^\prime 
    (f^{\prime} \to f^{\prime\prime})\). Με πράσινο χρώμα απεικονίζεται το 
    επίπεδο περιστροφής.

    \begin{figure}[h]%\kern-1.5ex
        \tikzset{cross/.style={
            cross out, draw=black, minimum size=2*(#1-\pgflinewidth),
            inner sep=0pt, outer sep=0pt}, cross/.default={1pt}}
        \centering
        \tdplotsetmaincoords{74}{170}
        \begin{tikzpicture}[scale = 2.5,tdplot_main_coords]
            % first half of second plane
            \draw[thin] (0, 0, 0) -- (0.866, 0.5, 0) -- (0.866, 0.5, -1) -- 
                (0, 0, -1) -- cycle;
            \fill[green!35, opacity=0.6] (0, 0, 0) -- (0.866, 0.5, 0) -- 
                (0.866, 0.5, -1) -- (0, 0, -1);
            
            % first plane
            \draw[thin] (1, 0, 0) -- (1, 1, 0) -- (-1, 1, 0) -- (-1, 0, 0) -- 
                cycle;
            \fill[blue!35, opacity=0.6] (1,0,0) -- (1,1,0) -- (-1,1,0) -- 
                (-1,0,0);

            %second half of second plane
            \draw[thin] (0.866, 0.5, 0) -- (0.866, 0.5, 1) -- (0, 0, 1)
                -- (0, 0, 0) -- cycle;
            \fill[green!35, opacity=0.6] (0.866, 0.5, 0) -- (0.866, 0.5, 1) -- 
                (0, 0, 1) -- (0, 0, 0);
            
            \draw (0,0,0) coordinate (O) node[above right]{\(O\)};
            % \draw[thick, -stealth] (0, 0, 0) -- (1, 0, 0) coordinate (a) 
            %     node[left] {\(X\)};
            % \draw[thick, -stealth] (0, 0, 0) -- (0, 1, 0) coordinate (b)
            %     node[right] {\(Y\)}; 
            \draw[thick, -stealth] (0, 0, 0) -- (0, 0, 1) coordinate (c) 
                node[above] {\(z^{\prime}\)};
            
            \draw[thick, -stealth] (0, 0, 0) -- (0.866, 0.5, 0) coordinate 
                (d) node[left] {\(x^{\prime}\)};
            \draw[thick, -stealth] (0, 0, 0) -- (-0.5, 0.866, 0) coordinate 
                (e) node[above] {\(y^{\prime}, y^{\prime\prime}\)};
            
            \draw[thick, -stealth] (0, 0, 0) -- (0.8138, 0.4698, -0.342) 
                coordinate (f) node[left] {\(x^{\prime\prime}\)};
            \draw[thick, -stealth] (0, 0, 0) -- (0.2962, 0.171, 0.9397) 
                coordinate (g) node[above] {\(z^{\prime\prime}\)};
                
            \draw[thick, draw=red, -stealth] (0, 0, 0) -- (-0.5, 0.866, 0) 
                node[below, red] {\(\dot{\theta}\)};

            \draw pic[thick, ->, "$\theta$", draw=red, text=red, angle radius=40, 
                angle eccentricity=1.4] {angle = d--O--f};
            \draw pic[thick, ->, "$\theta$", draw=red, text=red, angle radius=40,
                angle eccentricity=1.4] {angle = c--O--g};
        \end{tikzpicture}
        \caption{Περιστροφή Συστήματος προς άξονα \(Oy^{\prime}\).}
        \label{fig:rot_y}
    \end{figure}

    \item Περιστροφή του συστήματος αναφοράς \(f^{\prime\prime}\) κατά γωνία 
    διατοίχισης \(\phi\) (\tl{roll angle}) γύρω από τον τρέχων άξονα 
    \(Ox^{\prime\prime} (f^{\prime\prime} \to f)\). Με κίτρινο χρώμα 
    απεικονίζεται το επίπεδο περιστροφής.
    
    \begin{figure}[h]%\kern-1.5ex
        \tikzset{cross/.style={
            cross out, draw=black, minimum size=2*(#1-\pgflinewidth),
            inner sep=0pt, outer sep=0pt}, cross/.default={1pt}}
        \centering
        \tdplotsetmaincoords{85}{150}
        \begin{tikzpicture}[scale = 2.5, tdplot_main_coords]
            % first half of second plane
            \draw[thin] (0, 0, 0) -- (0.866, 0.5, 0) -- (0.866, 0.5, -1) -- 
                (0, 0, -1) -- cycle;
            \fill[green!35, opacity=0.4] (0, 0, 0) -- (0.866, 0.5, 0) -- 
                (0.866, 0.5, -1) -- (0, 0, -1);

            %first half of third plane
            \draw[thin] (0.5, -0.866, 0) -- (0.7962, -0.695, 0.9397) -- 
                (0.2962, 0.171, 0.9397) -- (0, 0, 0) -- cycle;
            \fill[yellow!35, opacity=0.6] (0.5, -0.866, 0) -- 
                (0.7962, -0.695, 0.9397) -- (0.2962, 0.171, 0.9397) -- 
                (0, 0, 0);

            %first plane
            \draw[thin] (1, 0, 0) -- (1, 1, 0) -- (-1, 1, 0) -- (-1, 0, 0) -- 
                cycle;
            \fill[blue!35, opacity=0.6] (1,0,0) -- (1,1,0) -- (-1,1,0) -- 
                (-1,0,0);

            % second half of second plane
            \draw[thin] (0.866, 0.5, 0) -- (0.866, 0.5, 1) -- (0, 0, 1)
                -- (0, 0, 0) -- cycle;
            \fill[green!35, opacity=0.4] (0.866, 0.5, 0) -- (0.866, 0.5, 1) -- 
                (0, 0, 1) -- (0, 0, 0);
            
            %second half of third plane
            \draw[thin] (-0.5, 0.866, 0) -- (-0.2038, 1.037, 0.9397) -- 
                (0.2962, 0.171, 0.9397) -- (0, 0, 0) -- cycle;
            \fill[yellow!35, opacity=0.6] (-0.5, 0.866, 0) -- 
                (-0.2038, 1.037, 0.9397) -- (0.2962, 0.171, 0.9397) -- 
                (0, 0, 0);


            \draw (0,0,0) coordinate (O) node[above right]{\(O\)};
            % \draw[thin, -stealth] (0, 0, 0) -- (1, 0, 0) coordinate (a) 
            %     node[left] {\(X\)};
            % \draw[thin, -stealth] (0, 0, 0) -- (0, 1, 0) coordinate (b)
            %     node[right] {\(Y\)}; 
            % \draw[thin, -stealth] (0, 0, 0) -- (0, 0, 1) coordinate (c) 
            %     node[above] {\(Z, z^{\prime}\)};
            
            
            % \draw[thin, -stealth] (0, 0, 0) -- (0.866, 0.5, 0) coordinate 
            %     (d) node[left] {\(x^{\prime}\)};
            \draw[thin, -stealth] (0, 0, 0) -- (-0.5, 0.866, 0) coordinate 
                (e) node[below] {\(y^{\prime\prime}\)};
            
            \draw[thick, -stealth] (0, 0, 0) -- (0.8138, 0.4698, -0.342) 
                coordinate (f) node[left] {\(x^{\prime\prime}, x\)};
            \draw[thick, -stealth] (0, 0, 0) -- (0.2962, 0.171, 0.9397) 
                coordinate (g) node[above] {\(z^{\prime\prime}\)};
            
            \draw[thick, -stealth] (0, 0, 0) -- (-0.3685, 0.8723, 0.3214) 
                coordinate (h) node[above left] {\(y\)};
            \draw[thick, -stealth] (0, 0, 0) -- (0.5065, -0.2849, 0.8138)
                coordinate (i) node[left] {\(z\)};
                
            \draw[thick, draw=red, -stealth] (0, 0, 0) -- 
                (0.8138, 0.4698, -0.342) node[below, red] {\(\dot{\psi}\)};

            \draw pic[thick, ->, "$\psi$", draw=red, text=red, angle radius=40, 
                angle eccentricity=1.4] {angle = e--O--h};
            \draw pic[thick, ->, "$\psi$", draw=red, text=red, angle radius=40,
                angle eccentricity=1.4] {angle = g--O--i};
        \end{tikzpicture}
        \caption{Περιστροφή Συστήματος προς άξονα \(Ox^{\prime\prime}\).}
        \label{fig:rot_x}
    \end{figure}

\end{itemize}

Οι αντιστοιχίες των συστημάτων αναφοράς με τα συστήματα που ορίστηκαν στην 
εισαγωγή του κεφαλαίου είναι \(F \to E\) και \(f \to B\).

Σύμφωνα με την ιδιότητα της πρόσθεσης, η γωνιακή ταχύτητα του στερεού σώματος 
προκύπτει ως

\begin{equation*}
    \bm{\omega}_{B/E} = {\bm{\omega}_{f/F}} = \bm{\omega}_{f/f^{\prime\prime}} + 
        \bm{\omega}_{f^{\prime\prime}/f^{\prime}} + 
        \bm{\omega}_{f^{\prime}/F} = \dot{\phi} \mathbf{e_{x}}
         + \dot{\theta} \mathbf{e_{y^{\prime}}}
         + \dot{\psi} \mathbf{e_{z^{\prime}}}.
\end{equation*}

Έχει ιδιαίτερη πρακτική σημασία η έκφραση των συνιστώσεων του διανύσματος 
\(\boldsymbol{\omega_{f/F}}\) στο σύστημα \(f\), το οποίο βρίσκεται στερεά 
δεμένο στο στερεό σώμα, λόγω της παροχής των μετρήσεων από τους αισθητήρες στο
κεντροβαρικό σύστημα \(B\).

\begin{equation} \label{knm:rot_coord}
    \begin{split}
    \omega_{x} = \bm{\omega}_{f/F} \cdot \mathbf{e_{x}} = 
        \dot{\phi}(\mathbf{e_{x}} \cdot \mathbf{e_{x}}) + 
        \dot{\theta}(\mathbf{e_{y^{\prime}}} \cdot \mathbf{e_{x}}) + 
        \dot{\psi}(\mathbf{e_{z^{\prime}}} \cdot \mathbf{e_{x}}) \\
    \omega_{y} = \bm{\omega}_{f/F} \cdot \mathbf{e_{y}} = 
        \dot{\phi}(\mathbf{e_{x}} \cdot \mathbf{e_{y}}) + 
        \dot{\theta}(\mathbf{e_{y^{\prime}}} \cdot \mathbf{e_{y}}) + 
        \dot{\psi}(\mathbf{e_{z^{\prime}}} \cdot \mathbf{e_{y}}) \\
    \omega_{z} = \bm{\omega}_{f/F} \cdot \mathbf{e_{z}} = 
        \dot{\phi}(\mathbf{e_{x}} \cdot \mathbf{e_{z}}) + 
        \dot{\theta}(\mathbf{e_{y^{\prime}}} \cdot \mathbf{e_{z}}) + 
        \dot{\psi}(\mathbf{e_{z^{\prime}}} \cdot \mathbf{e_{z}}).
    \end{split}
\end{equation}

Ο προσανατολισμός του συστήματος \(f^{\prime}\) ως προς το σύστημα \(f\) 
καθορίζεται πλήρως από τις τιμές των συνημιτόνων των γωνιών, τις οποίες 
σχηματίζουν οι άξονες \(Oi^{\prime}\) με \(i^{\prime} = \{x^{\prime}, y^{\prime},
z^{\prime}\}\) του συστήματος \(f^{\prime}\) ως προς τους άξονες \(Oi\) με 
\(i= \{x, y, z\}\). Δηλαδή, \(a_{i^{\prime},i} = e_{i^{\prime}} \cdot e_{i} = 
\cos(e_{i^{\prime}}, e_{i})\).

Άρα, δεδομένου των παραπάνω σχημάτων, εξάγονται τα εξής
\begin{gather*}
    \mathbf{e_{x}} \cdot \mathbf{e_{x}} = 1 \quad \text{λόγω παραλληλίας} \\
    \mathbf{e_{x}} \cdot \mathbf{e_{y}} = \mathbf{e_{x}} \cdot \mathbf{e_{z}} =
        \mathbf{e_{x}} \cdot \mathbf{e_{y^{\prime}}} = 0 \quad 
        \text{λόγω καθετότητας} \\
    \mathbf{e_{z^{\prime}}} \cdot \mathbf{e_{x}} = \cos\left(\pi/2 + 
        \theta\right) = -\sin\theta \\
    \mathbf{e_{y^{\prime}}} \cdot \mathbf{e_{z}} = \cos\left(\pi/2 + 
        \phi\right) = -\sin\phi \\
    \mathbf{e_{y^{\prime}}} \cdot \mathbf{e_{y}} = \cos\phi \\
    \mathbf{e_{z^{\prime}}} \cdot \mathbf{e_{y}} = (\mathbf{e_{x^{\prime}}} 
        \times \mathbf{e_{y^{\prime}}}) \cdot \mathbf{e_{y}} = 
        ( \mathbf{e_{y^{\prime}}} \times \mathbf{e_{y}}) \cdot 
        \mathbf{e_{x^{\prime}}} = \sin\phi (\mathbf{e_{x}} \cdot 
        \mathbf{e_{x^{\prime}}}) = \sin\phi \cos\theta \\
    \mathbf{e_{z^{\prime}}} \cdot \mathbf{e_z} = (\mathbf{e_{y^{\prime}}} 
        \times \mathbf{e_{x^{\prime}}}) \cdot \mathbf{e_z} = (\mathbf{e_z} 
        \times \mathbf{e_{y^{\prime}}}) \cdot \mathbf{e_{x^{\prime}}} = 
        (\mathbf{e_{x}} \cdot \mathbf{e_{x^{\prime}}}) \cos\phi = \cos\theta
        \cos\phi.
\end{gather*}

Οι συνιστώσες (\ref{knm:rot_coord}) λαμβάνουν την μορφή

\begin{gather*}
    \omega_{x} = \dot{\phi} - \dot{\psi} \sin\theta \\
    \omega_{y} = \dot{\theta} \cos\phi + \dot{\psi} \sin\phi \cos\theta \\
    \omega_{z} = -\dot{\theta} \sin\phi + \dot{\psi} \cos\phi \cos\theta.
\end{gather*}

Η επίλυση του παραπάνω συστήματος εξισώσεων ως προς τις χρονικές παραγώγους των 
γωνιών \tl{Euler} οδηγεί στο σύστημα μητρωϊκής μορφής

\begin{equation}
    \begin{pmatrix}
        {\dot{\phi}} \\
        {\dot{\theta}} \\
        {\dot{\psi}}
    \end{pmatrix} = 
    \begin{pmatrix}
        1 & \sin\phi \tan\theta & \cos\phi \tan\theta \\
        0 & \cos\phi & -\sin\phi \\
        0 & \sin\phi \sec\theta & \cos\phi \sec\theta 
    \end{pmatrix}
    \begin{pmatrix}
        {\omega_{x}} \\
        {\omega_{y}} \\
        {\omega_{z}}
    \end{pmatrix}
    \label{knm:rot_knm} \quad\text{ή}\quad  
    \begin{pmatrix}
        {\dot{\phi}} \\
        {\dot{\theta}} \\
        {\dot{\psi}}
    \end{pmatrix} = R_{eul}
    \begin{pmatrix}
        {\omega_{x}} \\
        {\omega_{y}} \\
        {\omega_{z}}
    \end{pmatrix}. 
\end{equation}

Οι παραπάνω σχέσεις αποτελούν σύστημα μη-γραμμικών κανονικών διαφορικών 
εξισώσεων και, συνεπώς, η αναλυτική τους λύση δεν είναι γνωστή. Η επίλυση του 
γίνεται μέσω αριθμητικών μεθόδων ολοκλήρωσης, έχοντας μετρήσει το διάνυσμα 
γωνιακής ταχύτητας του στερεού σώματος. Προκύπτει από την λύση ο 
προσανατολισμός του στερεού σώματος στον χώρο.

Το σύστημα παρουσιάζει πρόβλημα ορισμού για γωνίες \(\theta = \pi/2\), \(\theta 
= 3\pi/2\). Σε αυτήν την περίπτωση, οι γωνίες \(\phi\) και \(\psi\) περιγράφουν 
την ίδια περιστροφή και χάνεται ένας βαθμός ελευθερίας, καθιστώντας τον πίνακα 
ιδιάζων. Γίνεται αναγκαίος, λοιπόν, ο περιορισμός των γωνιών ως εξής

\begin{gather*}
    \phi \in (-\pi, \pi ] \\
    \theta \in \left(-\frac{\pi}{2}, \frac{\pi}{2} \right) \\
    \psi \in (-\pi, \pi].
\end{gather*}

Ορίζοντας το διάνυσμα \(\mathbf{e} = \left(\phi \quad \theta \quad \psi\right)
^\intercal\), το πεδίο ορισμού των γωνιών \tl{Euler} γράφεται στην εξής μορφή

\begin{equation*}
    \mathbf{e} \in U \quad \text{όπου} \quad U = (-\pi, \pi] \times 
    \left[-\frac{\pi}{2},\frac{\pi}{2} \right] \times (-\pi, \pi].
\end{equation*}

Το σύστημα (\ref{knm:rot_knm}) αποτελεί τη βασική κινηματική σχέση, η οποία 
συνδέει την παράγωγο των γωνιών \tl{Euler} με τις περιστροφικές ταχύτητες του 
αεροχήματος.

\subsection{Μητρώα Περιστροφής}
\noindent Κρίνεται σημαντικό να αναπτυχθεί μία μέθοδος για τον μετασχηματισμό
των συντεταγμένων ενός διανύσματος από ένα σύστημα αναφοράς σε ένα άλλο.

Ένα τυχαίο διάνυσμα \(\mathbf{r} \in \mathbb{R}^{3} \) παραμένει το ίδιο 
ανεξαρτήτως του συστήματος συντεταγμένων στο οποίο εκφράζεται, δηλαδή

\begin{equation*}
    \mathbf{r} = x\, \mathbf{e_{x}} + y\, \mathbf{e_{y}} + z\, \mathbf{e_{z}} = 
        X\, \mathbf{e_{X}} + Y\, \mathbf{e_{Y}} + Z\, \mathbf{e_{Z}},
\end{equation*}
όπου \( \mathbf{e_{x}}, \mathbf{e_{y}}\) και \(\mathbf{e_{z}}\) είναι τα 
διανύσματα βάσης του κεντροβαρικού συστήματος \(B\), ενώ \(\mathbf{e_{X}}, 
\mathbf{e_{Y}}\) και \(\mathbf{e_{Z}}\) είναι τα διανύσματα βάσης του 
αδρανειακού συστήματος \(E\). Για ευκολία κατανόησης, χρησιμοποιείται ο δείκτης 
\(B\) ή \(E\) για τον χαρακτηρισμό του συστήματος αναφοράς, που χρησιμοποιείται 
στην έκφραση του διανύσματος. Οι σχέσεις των συντεταγμένων των δύο συστημάτων, 
όπως αναφέρθηκε και στην ανάλυση των περιστροφικών κινηματικών σχέσεων, 
προκύπτουν με τον εξής τρόπο

\begin{align*}
    X = \mathbf{r}_{B} \cdot \mathbf{e_{X}} &= x\, \mathbf{e_{x}} \cdot 
        \mathbf{e_{X}} + y\, \mathbf{e_{y}} \cdot \mathbf{e_{X}} + z\, 
        \mathbf{e_{z}} \cdot \mathbf{e_{X}} \\
    Y = \mathbf{r}_{B} \cdot \mathbf{e_{Y}} &= x\, \mathbf{e_{x}} \cdot 
        \mathbf{e_{Y}} + y\, \mathbf{e_{y}} \cdot \mathbf{e_{Y}} + z\, 
        \mathbf{e_{z}} \cdot \mathbf{e_{Y}} \\
    Z = \mathbf{r}_{B} \cdot \mathbf{e_{Z}} &= x\, \mathbf{e_{x}} \cdot 
        \mathbf{e_{Z}} + y\, \mathbf{e_{y}} \cdot \mathbf{e_{Z}} + z\, 
        \mathbf{e_{z}} \cdot \mathbf{e_{Z}}.
\end{align*}

Σε μητρωική μορφή, λοιπόν, η έκφραση των συντεταγμένων του διανύσματος 
\(\mathbf{r}\) στο αδρανειακό σύστημα γίνεται 

\begin{equation*}
    \begin{pmatrix}
        X \\
        Y \\
        Z \\
    \end{pmatrix} = 
    \begin{pmatrix}
        \mathbf{e_{x}} \cdot \mathbf{e_{X}} & 
            \mathbf{e_{y}} \cdot \mathbf{e_{X}} &
            \mathbf{e_{z}} \cdot \mathbf{e_{X}} \\
        \mathbf{e_{x}} \cdot \mathbf{e_{Y}} & 
            \mathbf{e_{y}} \cdot \mathbf{e_{Y}} &
            \mathbf{e_{z}} \cdot \mathbf{e_{Y}} \\
        \mathbf{e_{x}} \cdot \mathbf{e_{Z}} & 
            \mathbf{e_{y}} \cdot \mathbf{e_{Z}} &
            \mathbf{e_{z}} \cdot \mathbf{e_{Z}}
    \end{pmatrix}
    \begin{pmatrix}
        x \\
        y \\
        z \\
    \end{pmatrix}
\end{equation*}
ή, σε συμπαγέστερη μορφή,
\begin{equation}
    \mathbf{r}_{E} = R\, \mathbf{r}_{B},
    \label{knm:rot_e_b}
\end{equation}
όπου \(R\) είναι το μητρώο περιστροφής, το οποίο καθορίζει τις συντεταγμένες 
του τυχαίου διανύσματος \(\mathbf{r}\) ως προς το αδρανειακό σύστημα αναφοράς, 
αν είναι γνωστές οι συντεταγμένες του \(\mathbf{r}\) στο κεντροβαρικό. 

Παρόμοια, για την αντίστροφη έκφραση 
\begin{align*}
    x = \mathbf{r}_{E} \cdot \mathbf{e_{x}} &= X\, \mathbf{e_{X}} \cdot 
        \mathbf{e_{x}} + Y\, \mathbf{e_{Y}} \cdot \mathbf{e_{x}} + Z\, 
        \mathbf{e_{Z}} \cdot \mathbf{e_{x}} \\
    y = \mathbf{r}_{E} \cdot \mathbf{e_{y}} &= X\, \mathbf{e_{X}} \cdot 
        \mathbf{e_{y}} + Y\, \mathbf{e_{Y}} \cdot \mathbf{e_{y}} + Z\, 
        \mathbf{e_{Z}} \cdot \mathbf{e_{y}} \\
    z = \mathbf{r}_{E} \cdot \mathbf{e_{z}} &= X\, \mathbf{e_{X}} \cdot 
        \mathbf{e_{z}} + Y\, \mathbf{e_{Y}} \cdot \mathbf{e_{z}} + Z\, 
        \mathbf{e_{Z}} \cdot \mathbf{e_{z}} \\
\end{align*}
καταλήγουμε στην μητρωϊκή μορφή
\begin{equation*}
    \begin{pmatrix}
        x \\
        y \\
        z \\
    \end{pmatrix} = 
    \begin{pmatrix}
        \mathbf{e_{X}} \cdot \mathbf{e_{x}} & 
            \mathbf{e_{Y}} \cdot \mathbf{e_{x}} &
            \mathbf{e_{Z}} \cdot \mathbf{e_{x}} \\
        \mathbf{e_{X}} \cdot \mathbf{e_{y}} & 
            \mathbf{e_{Y}} \cdot \mathbf{e_{y}} &
            \mathbf{e_{Z}} \cdot \mathbf{e_{y}} \\
        \mathbf{e_{X}} \cdot \mathbf{e_{z}} & 
            \mathbf{e_{Y}} \cdot \mathbf{e_{z}} &
            \mathbf{e_{Z}} \cdot \mathbf{e_{z}}
    \end{pmatrix}
    \begin{pmatrix}
        X \\
        Y \\
        Z \\
    \end{pmatrix}
\end{equation*}
ή σε συμπαγέστερη μορφή
\begin{equation}
    \mathbf{r}_{B} = R^\intercal\, \mathbf{r}_{E},
    \label{knm:rot_b_e}
\end{equation}
όπου πίνακας \(R^\intercal\) είναι ο ανάστροφος του \(R\), εφόσον ισχύει 
συμμετρία γινομένου στον διανυσματικό χώρο που ορίζονται τα συστήματα αναφοράς.
Από τις σχέσεις (\ref{knm:rot_e_b}) και (\ref{knm:rot_b_e}) προκύπτει ότι
\begin{equation*}
    \mathbf{r}_{E} = R (R^\intercal \, \mathbf{r}_{E}) = 
        (R R^\intercal) \mathbf{r}_{E}
\end{equation*}
και, επειδή το διάνυσμα \(\mathbf{r}\) είναι τυχαίο, θα ισχύει
\begin{equation}
    R R^\intercal = I_{3\times3}  \qquad \text{όπου} \quad \, I_{3x3} \, 
        \text{μοναδιαίος πίνακας}.
    \label{knm:ortho}
\end{equation}

Άρα το μητρώο περιστροφής \(R\) είναι ορθογώνιος πίνακας. Τα παραπάνω επιτρέπουν
την εύκολη μετατροπή των συντεταγμένων από ένα σύστημα αναφοράς σε ένα άλλο με 
απλή αναστροφή του πίνακα.

Η ιδιότητα (\ref{knm:ortho}) είναι η φυσική συνέπεια του ότι το μήκος του 
διανύσματος \(\mathbf{r}\) παραμένει αμετάβλητο κατά την περιστροφή των αξόνων.

Αναλύοντας τα συνημίτονα κατεύθυνσης που απαρτίζεται το μητρώο περιστροφής $R$, 
από το σύστημα αναφοράς $B$ στο σύστημα $E$, καταλήγουμε στην αναλυτική μορφή 
του

\begin{equation*}
    R = 
    \begin{pmatrix}
        \cos\psi \cos\theta & \cos\psi \sin\phi \sin\theta - \cos\phi \sin\psi
            & \sin\phi \sin\psi + \cos\phi \cos\psi \sin\theta \\
        \cos\theta \sin\psi & \cos\phi \cos\psi + \sin\phi \sin\psi \sin\theta
            & \cos\phi \sin\psi \sin\theta - \cos\psi \sin\phi \\
        -\sin\theta & \cos\theta \sin\phi & \cos\phi \cos\theta 
    \end{pmatrix}.
\end{equation*}

Για την ολοκλήρωση των κινηματικών εξισώσεων, απαιτείται η διατύπωση της σχέσης, 
που συνδέει την έκφραση της μεταφορικής ταχύτητας του κέντρου μάζας 
($\mathbf{v}_G$) του στερεού σώματος ανάμεσα στα δύο συστήματα αναφοράς. Έτσι, 
σύμφωνα με τον ορισμό του μητρώο περιστροφής
\begin{equation}
    \dot{\mathbf{p}} = R\, (\mathbf{v}_{G})_{B} = (\mathbf{v}_{G})_{E}
    \label{knm:trans_knm},
\end{equation}
ορίζοντας τα μεγέθη 
$\mathbf{p} = [{x_{E}} \quad {y_{E}} \quad {z_{E}}]^\intercal$, 
$(\mathbf{v}_{G})_{E} = [\dot{x_{E}} \quad \dot{y_{E}} \quad 
    \dot{z_{E}}]^\intercal$ και 
$(\mathbf{v}_{G})_{B} = [u \quad v \quad w]^\intercal$
προκύπτει η μεταφορική κινηματική σχέση.