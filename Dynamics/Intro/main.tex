\section{Εισαγωγή}
\noindent Στόχος του κεφαλαίου είναι η ανάπτυξη των μαθηματικών εξισώσεων, που 
περιγράφουν την κίνηση του τρικόπτερου. Συγκεκριμένα, γίνεται ο ορισμός των 
κινηματικών εξισώσεων του οχήματος, μοντελοποίηση των ασκούμενων δυνάμεων και εν 
τέλει η σύνθεση των εξισώσεων κίνησης. 

Το τρικόπτερο διαθέτει 5 ενεργοποιητές, 3 από τους οποίους είναι οι έλικες, ενώ
οι υπόλοιποι είναι οι 2 σερβομηχανισμοί. Οι έλικες έχουν στόχο την παραγωγή 
δυνάμεων ώθησης, με παράπλευρη συνέπεια την παραγωγή ροπών αντίστασης. Οι 
σερβομηχανισμοί αποτελούν τους μηχανισμούς κεκλιμένου στροφέιου και επιτυγχάνουν
την περιστροφή των διανυσμάτων των δυνάμεων ώθησης και ροπών αντίστασης, για 
τους δύο εμπρόσθιους κινητήρες. 

%EDIT
Στην ανάλυση, που ακολουθεί, έχουν γίνει κάποιες παραδοχές. Το τρικόπτερο 
θεωρείται ως ένα στερεό απαραμόρφωτο σώμα. Έτσι, οι εσωτερικές δυνάμεις και 
ροπές, που ασκούνται τόσο στους κινητήρες όσο κι από το μηχανισμό κεκλιμένου 
στροφείου δεν λήφθηκαν υπόψιν. Ακόμη, παραλήφθηκαν οι αεροδυναμικές δυνάμεις 
κατά τη λειτουργία αιώρησης καθώς δεν αναπτύσσονται σημαντικές ταχύτητες.
% Το τρικόπτερο, όπως αναφέρθηκε στην εισαγωγή, διαθέτει τρεις έλικες για την 
% παραγωγή ώθησης, ενω με τους δύο μπροστινούς μηχανισμούς κεκλιμένου στροφείου, 
% επιτρέπεται η αλλαγή προσανατολισμού των αντίστοιχων ελίκων. Ο έλεγχος της 
% κίνησης, λοιπόν, γίνεται με την μεταβολή των γωνιακών ταχυτήτων και ως 
% συνέπεια, των παραγόμενων δυνάμεων και ροπών, ενώ με την αλλαγή του 
% προσανατολισμού των μηχανισμών, επιτυγχάνεται η αλλαγή διεύθυνσης των 
% προκύπτουν σών δυνάμεων και ροπών.