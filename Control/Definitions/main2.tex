
    Έστω τώρα οι γραμμικοί χώροι $E,F$ και ένας ομομορφισμός  $f:E\rightarrow F$
    . Τότε ο \textbf{πυρήνας} της $f$, που συμβολίζεται με $\ker f$, είναι το 
    σύνολο $\{x\in E |\,f(x) = 0\}$.


\begin{nameddefn}{Ορισμοί}
    Υποθέτουμε ότι $S$ είναι ένα διατεταγμένο σύνολο, ότι $E\subset S$ και ότι 
    το $E$ είναι άνω φραγμένο. Επίσης, υποθέτουμε ότι υπάρχει $\alpha \in S$ με
    τις ακόλουθες ιδιότητες:
    \begin{enumerate}
        \item Το $\alpha$ είναι άνω φράγμα του $E$.
        \item Έαν το $gamma$ είναι στοιχείο στο $S$ με $\gamma < \alpha$, τότε 
        το $\gamma$ δεν είναι άνω φράγμα του $E$. 
    \end{enumerate}
    Τότε, και μόνον τότε, το $\alpha$ ονομάζεται το ελάχιστο άνω φράγμα του $E$
    ή αλλιώς το $supremum$  του $E$ και γράφουμε 
    \begin{equation*}
        \alpha = \sup E
    \end{equation*} 
    Το μέγιστο κάτω φράγμα ή αλλιώς το $infinum$ ενός κάτω φραγμένου συνόλου $E$
    ορίζεται αναλόγως η πρόταση
    \begin{equation*}
        \alpha = \inf E
    \end{equation*}
    δηλώνει ότι το $\alpha$ είναι κάτω φράγμα του E και δεν υπάρχει κάτω φράγμα
    $\beta$ του $E$ με $\beta > \alpha$.
\end{nameddefn}


\begin{nameddefn}{Ορισμοί}
    Έστω οι γραμμικοί χώροι $V$ και $W$ επί του $\realf$. Μια απεικόνιση $f:V
    \rightarrow W$ ονομάζεται γραμμική ή ομοομορφισμός, αν για όλα τα $x, y \in
    V, \lambda \in \realf$ έχουμε
    \begin{align*}
        f(x+y) =& f(x) + f(y)\\
        f(\lambda x) =& \lambda f(x)
    \end{align*}
\end{nameddefn}