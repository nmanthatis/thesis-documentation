Έστω τα σύνολα $S, \, T$, με $S \subset T$, τότε \tl{\textbf{Infimum} (inf)} του $S$ είναι το μεγαλύτερο στοιχείο του $T$ το οποίο είναι μικρότερο ή ίσο όλων των στοιχείων του $S$. Αντίστοιχα το \tl{\textbf{Supremum} (sup)} είναι το μικρότερο στοιχείο του $T$ το οποίο είναι μεγαλύτερο ή ίσο όλων των στοιχείων του $S$.

Έστω η απεικόνιση $f:X \rightarrow Y$ και $A \subset X, B\subset Y$. Τότε το σύνολο $f(A):=\left\{f(x)\,|\,x \in A \right\}$ ονομάζεται η \textbf{εικόνα} του $A$ και συμβολίζεται με $im\,A$. Επίσης το σύνολο $f^{-1}(B):=\left\{x\,|\,f(x) \in B\right\}$ ονομάζεται η \textbf{αντίστροφη εικόνα} του $B$. \textbf{Πυρήνας} μιας απεικόνισης ονομάζεται η αντίστροφη εικόνα του μηδενός δηλαδή, $\ker f :=\left\{x \in X\,|\,f(x) = 0\right\}$.

\textbf{Πρόταση} Για μια γραμμική απεικόνιση $T:V\rightarrow W$ ισχύει $\dim im T^{*} = \dim(\ker T)^{\bot}$.\\ \textbf{Απόδειξη} Έστω $v\in im\,T^*$ και $w\in \ker T$. Η εικόνα του $T^*$ είναι της μορφής $T^*v$ για $v \in V$, έτσι αρκεί να δειχθεί ότι $\langle T^*v, w \rangle = 0$. Πράγματι ισχύει, $\langle T^*v, w \rangle = T^*\langle v, w \rangle = \langle v, Tw \rangle = \langle v, 0 \rangle = 0$.s