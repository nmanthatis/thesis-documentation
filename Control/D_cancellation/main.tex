\section{Εξασθένιση Διαταραχών}

\noindent Στο σύστημα (\ref{ctl:knt})-(\ref{ctl:knm}) προστίθεται η απεικόνιση
$\overline{w} \in \class 1(\realf\times\euclr n\times V,\realf)$, η οποία 
αντιπροσωπεύει το σύνολο των διαταραχών που επιδρούν σε αυτό. Προκύπτει το 
σύστημα
\begin{gather}
    \dot{x} = f_{0}(x, y) + \sum_{i}^{m}{d_{i}f_{i}}(x,y) + \overline{w}(t,x,y)
    \label{ctl:knt_} \\
    \dot{y} = A(y)x.
    \label{ctl:knm_}   
\end{gather}

Επίσης, στον όρο της δράσης ελέγχου (\ref{ctl:ctl}) προστίθεται μία ακόμη δράση,
$\widehat{v}: \realf\times\euclr n\times V \rightarrow \euclr n$. Επιπλέον,
έστω $\phi:[t_0,t_1)\rightarrow \euclr n \times V$ ολοκληρωτική καμπύλη του
διανυσματικού πεδίου κλειστού βρόχου που διέρχεται από το $\left(t_0,(x_0,y_0)
\right)$. Θέτουμε $y_{r0} = y_r(t_0),\quad v_0 = \dot{y}_r(t_0),\quad \phi_1 = 
\pi_1 \circ \phi$ και $\phi_2 = \pi_2 \circ \phi$. Για τις καμπύλες $\phi_1$, 
$\phi_2$ ισχύει

\begin{gather}
    \dot{\left(\phi_1 - A(\phi_2)^{-1}\left(K_1(\phi_2 - y_r) +\dot{y}_r\right)
    \right)} = K_2\left(\phi_1 - A(\phi_2)^{-1}\left(K_1(\phi_2 - y_r) + 
    \dot{y}_r\right)\right) + \widehat{v} + \overline{w}
    \label{ctl:dphi1}\\
    \dot{\phi}_2 = A(\phi_2)\phi_1.
    \label{ctl:dphi2}
\end{gather}

Από την πρώτη έπεται ότι

\begin{equation}
    \begin{split}5717652
        \left(\phi_1 - A(\phi_2)^{-1}\left(K_1(\phi_2 - y_r) +\dot{y}_r\right)
        \right) =& e^{(t-t_0)K_2} \left(x_0 - A(y_0)^{-1}\left(K_1(y_0 - y_{r0})
         + v_0\right)\right) +\\&\int_{t_0}^{t}{e^{(t-s)K_2}\,\widehat{v}(s)} ds
         + \int_{t_0}^{t}{e^{(t-s)K_2}\,\overline{w}(s)} ds.
    \end{split}
    \label{ctl:phi1}
\end{equation}

Οι πρώτοι όροι στο δεξί μέλος είναι η ολοκληρωτική καμπύλη $\phi_3$ του
\begin{equation*}
    (t,x) \mapsto K_2x+\widehat{v}
\end{equation*}
η οποία διέρχεται από το $\left(t_0,\quad x_0 - A(y_0)^{-1}\left(K_1 (y_0 - 
y_{r0})+v_0\right)\right).$

Αντικαθιστώντας την (\ref{ctl:phi1}) στην (\ref{ctl:dphi2}) προκύπτει
\begin{equation}
    \dot{(\phi_2 - y_r)} = K_1 (\phi_2 - y_r) + A(\phi_2) \phi_3 + 
    A(\phi_2)\int_{t_0}^{t}{e^{(t-s)K_2}\,\overline{w}(s)} ds.    
\end{equation}

Καταλήγουμε ότι $t \mapsto (\phi_2(t) - y_r(t),\quad \phi_3(t))$ είναι η 
ολοκληρωτική καμπύλη του διανυσματικού πεδίου
\begin{equation*}
    \left(t, (y,z)\right) \mapsto \left(K_1y + A(y + y_r)z + A(y + y_r)
    \int_{t_0}^{t}{e^{(t-s)K_2}\,\overline{w}(s)} ds,\quad K_2z + 
    \widehat{v}\right)
\end{equation*}
που διέρχεται από το $\left(t_0,\left(y_0 - y_{r0},\quad x_0 - A(y_0)^{-1}\left(
K_1(y_0 - y_{r0}) + v_0\right)\right)\right)$.

Επιθυμούμε να γραμμικοποιήσουμε τον μη-γραμμικό όρο $A(y + y_r)z$. Στο 
$A(y + y_r)$, η γωνία $\psi$ εισέρχεται γραμμικά. Επιπλέον, για τον έλεγχο της 
αιώρησης, στόχος είναι η διατήρηση των αποκλίσεων γωνιών $\phi$ και $\theta$ 
απο τις τροχιές αναφοράς σε μηδενικές τιμές. Έτσι επιλέγεται ως σημείο 
γραμμικοποίησης το $(0, 0)$, το οποίο αποτελεί κέντρου του ανοιχτού συνόλου $V$. 
Θέτουμε $g:V \times \euclr n \rightarrow \euclr n$ με $g(y, z) = A(y)z$ με στόχο την 
γραμμικοποίηση. Ισχύει ότι
\begin{equation*}
    g^{\prime}(y, z)(h_1, h_2) = A(y) h_2 + (A^{\prime}(y)h_1)z.
\end{equation*}
Άρα,
\begin{equation*}
    g(y, z) = g(0, 0) + g^{\prime}(0, 0)(h_1, h_2) + \smallo((h_1, h_2)) = 
    A(0)h_2.
\end{equation*}

Πλέον το διανυσματικό πεδίο γράφεται ως
\begin{equation*} 
    \left(t, (y,z)\right) \mapsto \left(K_1y + A(0)z + w,\quad K_2z + 
    \widehat{v}\right)
\end{equation*}
και καταλήγουμε σε μητρωική μορφή
\begin{equation}
    \dot{x_1} = A_1 x_1 + Ew + B_1 u_1
    \label{ctl:linsys}
\end{equation}
με 
\begin{equation*}
    A_1 =
    \begin{pmatrix}
        K_1 & A(0) \\
        0 & K_2
    \end{pmatrix} \qquad
    B_1 = 
    \begin{pmatrix}
        0 \\
        I
    \end{pmatrix} \qquad
    E = 
    \begin{pmatrix}
        I \\
        0
    \end{pmatrix}
\end{equation*}
και $x_1 = \left(\phi_2 - y_r, \, \phi_3\right)^{\intercal}, \, w = 
A(0)\int_{t_0}^{t}{e^{(t-s)K_2}\,\overline{w}(s)} ds$ και $u_1 = \widehat{v}.$

Διαθέτοντας πλέον γραμμική έκφραση του συστήματος, στοχεύουμε στην εξασθένιση 
των διαταραχών του. Είναι σημαντικό να βρεθεί τρόπος ποσοτικοποίησης της δράσης
των διαταραχών επί του συστήματος.  

% \begin{nameddefn}{Ορισμοί}
%     Μια ακολουθία $(x_n)_1^\infty$ ονομάζεται \textbf{ακολουθία \tl{Cauchy}} αν 
%     για $\epsilon > 0$, υπάρχει $N$ τέτοιο ώστε 
%     \begin{equation*}
%         \norm{x_n - x_m} < \epsilon\quad \forall \, n,m \geq N.
%     \end{equation*}
%     % Δηλαδή, μια σειρά λέγεται σειρά \tl{Cauchy}  αν τα στοιχεία της σειράς 
%     % έρχονται όλο και πιο κοντά όσο κινούμαστε προς το τέλος της σειράς.
% \end{nameddefn}
% \begin{nameddefn}{Ορισμοί}
%     Ένας γραμμικός χώρος με νόρμα στον οποίο οι σειρές \tl{Cauchy} συγκλίνουν 
%     σε ένα σημείο ονομάζεται \textbf{χώρος \tl{Banach}}. (Παραδείγματα χώρων 
%     \tl{Banach} είναι οι γραμμικοί χώροι με νόρμα πεπερασμένης διάστασης 
%      $\realf^n$.)
% \end{nameddefn}

\begin{nameddefn}{Συμβολισμοί}
    Έστω $X$ χώρος \tl{Banach}. Δηλαδή, ένας χώρος με νόρμα, ο οποίος είναι 
    πλήρης μετρικός χώρος ως προς τη μετρική $d$ που επάγεται από τη νόρμα του.
    Αυτό σημαίνει ότι κάθε \tl{Cauchy} (ως προς $d$) ακολουθία σημείων του 
    χώρου συγκλίνει (ως προς $d$) σε ένα σημείο του χώρου. Παραδείγματα χώρων 
    \tl{Banach} είναι οι γραμμικοί χώροι με νόρμα πεπερασμένης διάστασης, διότι 
    είναι ισομορφικοί με κάποιο $\euclr n$.
    
    Η πληρότητα του χώρου μας επιτρέπει να ορίσουμε ένα ολοκλήρωμα για τμηματικά
    συνεχείς (και άρα συνεχείς) συναρτήσεις με τιμές στο $X$, το οποίο έχει 
    παρόμοιες ιδιότητες με το ολοκλήρωμα \tl{Riemann} πραγματικών συναρτήσεων. 
    
    Έστω $X$ χώρος \tl{Banach}, τότε αν $a, b \in \realf$ με $a <b$, 
    συμβολίζουμε με $\class{} \left([a,b],X\right)$ το σύνολο των συνεχών 
    απεικονίσεων του $[a, b]$ στο $X$.
    
    Επίσης, συμβολίζουμε με $L^2\left([a, \infty),X\right)$ το σύνολο των 
    συνεχών απεικονίσεων $f:[\alpha,\infty)\rightarrow X$ για τις οποίες ισχύει
    \begin{equation*}
        \int_{a}^{\infty} ||f(t)||^2 dt = \lim_{b\rightarrow \infty}\int_{a}^{b} 
        ||f(t)||^2 dt < \infty.
    \end{equation*}
    Με όμοιο τρόπο ορίζεται και το σύνολο $L^2\left((-\infty,a],X\right)$.

    Για περισσότερες πληροφορίες στα \cite{Amann2005}, \cite{Lang1997}
\end{nameddefn}

(α) Κάθε χώρος $L^2([0, \infty), X)$ είναι γραμμικός χώρος: Πράγματι, αν $f, g \in 
L^2([0, \infty), X)$, τότε η $f + g$ είναι συνεχής και επειδή $2\gamma\delta \leq
\gamma^2 + \delta^2$ για κάθε $\gamma, \delta \in \realf$ έχουμε ότι
\begin{equation*}
    \int_{a}^{t} \norm{f + g}^2 \leq \int_{a}^{t} \left( \norm{f}^2 + 2\norm{f}
    \norm{g} + \norm{g}^2 \right) \leq 2 \int_{a}^{t} \norm{f}^2 + 2 
    \int_{a}^{t} \norm{g}^2 \leq 2 \int_{a}^{\infty} \norm{f}^2 + 2 
    \int_{a}^{\infty} \norm{g}^2
\end{equation*}
για $t\in\realf$ με $a < t.$

Παίρνοντας το όριο καθώς $t \rightarrow \infty$ προκύπτει ότι
\begin{equation*}
    \int_{a}^{\infty} \norm{f+g}^2 \leq 2 \int_{a}^{\infty} \norm{f}^2 + 2 
    \int_{a}^{\infty} \norm{g}^2
\end{equation*}
και άρα $f+g\in L^2([a, \infty), X)$.

Προφανώς, αν $c\in \realf$, τότε $cf\in L^2([0, \infty), X)$. Επομένως, ο 
$L^2([0, \infty), X)$, με τις κατά σημείο πράξεις της πρόσθεσης και του βαθμωτού 
πολλαπλασιασμού, είναι γραμμικός χώρος.

(β) Η απεικόνιση
\begin{equation*}
    \norm{\cdot}:L^2([0, \infty), X) \rightarrow [0, +\infty)\quad \text{με}
    \quad \norm{f} = \left(\int_{a}^{\infty} \norm{f(t)}^2 dt\right)^{1/2} 
\end{equation*} 
είναι νόρμα στο $L^2([a, \infty), X)$. Προφανώς $\norm{af} = |a|\norm{f}$ για 
κάθε $a \in \realf$. Επίσης, αν $g:[c, d] \rightarrow [0, \infty)$ είναι μία 
ολοκληρώσιμη συνάρτηση, η οποία είναι συνεχής σε ένα σημείο $x \in[c, d]$ με 
$g(x) > 0$, τότε $\int_{c}^{d} g >0$. Η $\norm{\cdot} \circ f$ είναι συνεχής 
συνάρτηση ως σύνθεση συνεχών συναρτήσεων και έτσι ολοκληρώσιμη. Έπεται άμεσα ότι
 $\norm{f} = 0$, αν και μόνο αν $f = 0$.

Απομένει να δείξουμε την τριγωνική ιδιότητα. Πρώτα, θα αποδειχτεί ότι
\begin{equation*}
    \int_{a}^{\infty} \norm{f(t)} \norm{g(t)} dt \leq \norm{f} \norm{g} \quad 
    \text{για} \quad f, g \in L^2([a, \infty), X).
\end{equation*}

Χωρίς βλάβη της γενικότητας, αρκεί να δείξουμε ότι η παραπάνω ανισότητα ισχύει 
για $f, g \in L^2([a, \infty), X)$ με $\norm{f} = \norm{g} = 1$. Για $t\in\realf$
με $a < t$, έχουμε ότι
\begin{equation*}
    \begin{split}
        \int_{a}^{t} \norm{f(s)} \norm{g(s)} ds \leq \frac{1}{2} \left(
        \int_{a}^{t} \norm{f(s)}^2 ds + \int_{a}^{t}\norm{g(s)}^2 ds \right) \\
        \leq \frac{1}{2} \left(\int_{a}^{\infty} \norm{f(s)}^2 ds + 
        \int_{a}^{\infty}\norm{g(s)}^2 ds \right) = 1 = \norm{f} \norm{g}.    
    \end{split}
\end{equation*}
Το αποτέλεσμα προκύπτει παίρνοντας το όριο καθώς $t \rightarrow \infty$

Τέλος, για $f, g \in L^2([a, \infty), X)$ με $f + g \neq 0$, ισχύει
\begin{equation*}
    \begin{split}
        \norm{f + g}^2 = \int_{0}^{\infty} \norm{f(t) + g(t)}^2 dt = 
        \int_{0}^{\infty} \norm{f(t) + g(t)}\norm{f(t) + g(t)} dt \\ 
        \leq \int_{0}^{\infty} \norm{f(t)}\norm{f(t) + g(t)} dt + 
        \int_{0}^{\infty} \norm{g(t)}\norm{f(t) + g(t)} dt,
    \end{split} 
\end{equation*}
το οποίο προκύπτει από την τριγωνική ιδιότητα της νόρμας $\norm{\cdot} :X
\rightarrow [0,\infty)$.
Επομένως
\begin{equation*}
        \norm{f + g}^2 \leq \norm{f}\norm{f+g} + \norm{g}\norm{f+g}.
\end{equation*}
Διαιρώντας με $\norm{f+g}$ καταλήγουμε ότι 
\begin{equation*}
    \norm{f+g} \leq \norm{f} + \norm{g}.    
\end{equation*}

(γ) Αν ο $\left(X,\langle\cdot,\cdot\rangle\right)$ είναι χώρος \tl{Hilbert} 
(δηλαδή, ο $X$ είναι χώρος με εσωτερικό γινόμενο, ο οποίος είναι χώρος 
\tl{Banach} με τη νόρμα που επάγεται από το $\langle\cdot,\cdot\rangle$), τότε η
απεικόνιση
\begin{equation*}
    \langle\cdot,\cdot\rangle : L^2\left([a,\infty),X\right)^2 \rightarrow 
    \realf\quad\text{με}\quad \langle f, g \rangle = \int_o^{\infty}\langle f(t)
    ,g(t) \rangle
\end{equation*}
είναι εσωτερικό γινόμενο στο $L^2([a,\infty,X)$. Από την ανισότητα \tl{Cauchy-
Schwarz} και την πρώτη ανισότητα που αποδείχθηκε στο (β), προκύπτει ότι
\begin{equation*}
    \int_a^{\infty}\langle f(t),g(t) \rangle \leq \int_a^{\infty} ||f(t)||\,
    ||g(t)|| \leq ||f(t)||\,||g(t)|| < \infty.    
\end{equation*}
H θετικότητα της $\langle\cdot,\cdot\rangle$ έχει αποδειχθεί στο (β). Οι 
υπόλοιπες ιδιότητες (διγραμμικότητα, συμμετρία) είναι προφανείς.

\begin{namedthrm}{Θεώρημα \tl{Lyapunov}}
    Έστω $X$ γραμμικός χώρος πεπερασμένης διάστασης με εσωτερικό γινόμενο και 
    $A \in \mathcal{L}(X)$. Τότε όλες οι ιδιοτιμές του $A$ (στο 
    $\mathbb{C})$ έχουν αρνητικό πραγματικό μέρος, αν και μόνο αν για κάθε $B 
    \in \mathcal{L}(X)$ υπάρχει μοναδική λύση $\Lambda$ της εξίσωσης
    $A^{*} \circ \Lambda + \Lambda \circ A = B$ και αν $B < 0$, τότε $ \Lambda 
    > 0$.

    Επιπλέον, ισχύει ότι $\Lambda= -\int_{0}^{\infty} e^{t A^{*}} \circ 
    B \circ 
    e^{t A}$.
\end{namedthrm}
\begin{proof}
    Βλ. \cite{lax_07:linear}, \cite{Sontag1998}
\end{proof}
Υποθέτουμε ότι $A \in \mathcal{L}(\euclr n)$ και όλες οι ιδιοτιμές του $A$ (στο
$\mathbb{C}$) έχουν αρνητικό πραγματικό μέρος. Επιπλέον υποθέτουμε ότι $w \in 
L^2([0, \infty), \euclr n)$. Θεωρούμε ένα γραμμικό σύστημα 
\begin{equation}
    \dot{x} = Ax + Bw \quad \text{με} \quad x(0) = z, \quad y = Cx + Dw.
    \label{ctl:lin}    
\end{equation}
Λύση της διαφορικής εξίσωσης είναι η
\[x(t) = e^{tA}z + \int_{0}^t e^{(t - s)A} B w(s) ds.\]
Έτσι για την έξοδο του συστήματος ισχύει 
\begin{equation}
    y(t) = Ce^{tA}z + C\int_{0}^t e^{(t - s)A} B w(s) ds + Dw(t)
    \label{ctl:out}    
\end{equation} 
\[\text{με}\quad x \in \euclr p, y \in \euclr q, w \in \euclr r.\]
Εφόσον όλες οι ιδιοτιμές της $A$ βρίσκονται στο αριστερό ημιεπίπεδο, σύμφωνα με 
το θεώρημα \tl{Lyapunov} υπάρχει μοναδική $\Lambda >0$ με $\Lambda = \Lambda ^*$
, ώστε 
\begin{equation}
    A^{*} \circ \Lambda + \Lambda \circ A = -I.
    \label{lyap}
\end{equation}
Θέτουμε $g(t) = \int_{0}^t e^{(t - s)A} B w(s) ds$ και $\alpha(t) = \langle
\Lambda g(t),g(t)\rangle$. Προφανώς ισχύει $\dot{g} = Ag +Bw$. Τότε η $\alpha$
είναι διαφορίσιμη ως σύνθεση διαφορίσιμων συναρτήσεων και ισχύει ότι
\begin{align*}
    \dot{\alpha} &= \langle\Lambda \dot g,g\rangle + \langle\Lambda g,\dot g
    \rangle = \langle \Lambda \left(A g + Bw\right),g \rangle + \langle 
    \Lambda g,A g + Bw\rangle \\
    & = \langle \Lambda A g,g \rangle + \langle \Lambda Bw,g \rangle
          + \langle \Lambda g,A g \rangle + \langle \Lambda g, Bw\rangle\\
    & = \langle \Lambda A g,g \rangle + \langle A ^*\Lambda g,g \rangle
    + \langle w, B^* \Lambda^*g\rangle + \langle B^*\Lambda g,w \rangle\\
    & = \langle \left(A^* \Lambda + \Lambda A\right) g, g\rangle 
    + 2\langle B^* \Lambda g,w\rangle.
\end{align*}
Από τη σχέση (\ref{lyap}) και επιλέγοντας τυχαία σταθερά \tl{c} η εξίσωση 
παίρνει τη μορφή
\begin{align*}
\dot{\alpha}& = -\norm g ^2 + c^2\langle w, w\rangle + 2\langle B^* \Lambda g,w
    \rangle    - c^2\langle w, w\rangle\\
    & = -\norm g ^2 + c^2\langle w, w\rangle -c^2\langle w - 
    c^{-2}B^* \Lambda g,w -c^{-2} B^* \Lambda g\rangle  
    +c^{-2}\langle B^* \Lambda g,B^* \Lambda g\rangle\\
    & = -\norm g ^2 + c^2\norm{w}^2 -c^2\norm{w - c^{-2}B^* \Lambda g}^2  
    +c^{-2} \norm{B^* \Lambda g}^2.
\end{align*}
Επειδή ο τρίτος όρος της παραπάνω ισότητας είναι μη θετικός, έπεται ότι
\[\dot{\alpha}  \le   (c^{-2}\norm{B^*}^2 \norm{\Lambda}^2  -1) \norm g ^2+ 
    c^2 \norm w ^2. \]
Ολοκληρώνοντας την παραπάνω ανισότητα, παίρνουμε
\[   a(t) - a(0) \le \int_{0}^{t} \left((c^{-2}\norm{B^*}^2 \norm{\Lambda}^2  -1) 
    \norm g ^2+ c^2 \norm w ^2 \right).\]
Από το γεγονός ότι $\Lambda > 0$, συμπεραίνεται ότι $a(t) \ge 0$ για  
$t\in[0,\infty)$. Το $a(0) = 0$, αφού $g(0) = 0$. Άρα,
\begin{align*}
    0 &\le (c^{-2}\norm{B^*}^2 \norm{\Lambda}^2  -1) \int_{0}^{t} \norm g ^2
    + c^2 \int_{0}^{t} \norm w ^2 \\
    (1& - c^{-2}\norm{B^*}^2 \norm{\Lambda}^2) \int_{0}^{t} \norm g ^2 \le c^2 
    \int_{0}^{t} \norm w ^2 \le c^2 \int_0^\infty \norm{w}^2.
\end{align*}
    % 0 &\le (c^{-2}\norm{B^*}^2 \norm{\Lambda}^2  -1) \int_{0}^{t} \norm g ^2
    % + c^2 \int_{0}^{t} \norm w ^2 \\
    % (1& - c^{-2}\norm{B^*}^2 \norm{\Lambda}^2) \int_{0}^{t} \norm g ^2 \le c^2 
    % \int_{0}^{t} \norm w ^2 \le \int_0^\infty \norm{w}^2\\
    % (1 -& c^{-2}\norm{B^*}^2 \norm{\Lambda}^2) \int_{0}^{t} \norm g ^2 \le c^2
    %  \lim_{t\rightarrow\infty}\int_{0}^{t} \norm 
    % w ^2 < \infty \\ 
    % &(1 - c^{-2}\norm{B^*}^2 \norm{\Lambda}^2)\int_{0}^{t} \norm{g}^2 \le c^2
    %  \norm{w}^2
% \end{align*} 
Επιλέγοντας αρκούντως μεγάλο $c$ ώστε $(1 - c^{-2}\norm{B^*}^2\norm{\Lambda}^2)
 >0$ και παίρνοντας το όριο, υπάρχει $\gamma>0$ τέτοιο ώστε
 \[\norm g^2 = \int_0^\infty \norm g^2 \le \gamma^2 \int_0^\infty \norm{w}^2 = 
 \gamma^2 \norm{w}^2 < \infty.\]

Έτσι, έχουμε δείξει πως και η απεικόνιση $g$ ανήκει στο χώρο $L^2([0, \infty), 
\euclr n)$. Στην διαδικασία που προηγήθηκε, είναι σημαντικό να τονισθεί, ότι η 
επιλογή της $c$ δεν γίνεται βάσει υποθέσεων για τις $w$, αλλά εξαρτάται μόνο από
ενδογενή χαρακτηριστικά του συστήματος. 

% (\ref{ctl:knt_})(\ref{ctl:linsys}). 
Ακολουθώντας την ίδια διαδικασία μπορούμε να δείξουμε ότι και η απεικόνιση $w$,
όπως ορίζεται στην (\ref{ctl:linsys}), ανήκει στο χώρο $L^2([0, \infty), 
\euclr n)$ καθώς η απεικόνιση $\overline{w}$, όπως έχουμε υποθέσει εξαρχής 
(\ref{ctl:knt_}), ανήκει στο χώρο $L^2([0, \infty), \euclr n)$.

Στην μόνιμη κατάσταση, η έξοδος του συστήματος επηρεάζεται μόνο από τους όρους 
που εμπεριέχουν τις διαταραχές. Αυτό συμβαίνει, διότι ο πρώτος όρος της 
(\ref{ctl:out}) εξασθενεί στην μόνιμη κατάσταση λόγω αρνητικών ιδιοτιμών του 
$A$. 

Ορίζεται η απεικόνιση $T\colon L^2 \to L^2$ με $w 
\mapsto C g + D w = y$ η οποία είναι γραμμική ως πρόσθεση γραμμικών απεικονίσεων
. 

Από την τριγωνική ανισότητα της νόρμας στo $L^2$ ισχύει
\[\norm{T w} \le \norm C \norm g + \norm D + \norm w \le \left(\norm C 
\gamma + \norm D\right)\norm w\]
άρα η απεικόνιση $T$ είναι γραμμική και συνεχής. Έτσι η έξοδος του συστήματος 
είναι φραγμένη αν και η είσοδος είναι φραγμένη.

\begin{namedthrm}{Πρόταση}{\tl{Bounded Real Lemma}}
    
    Έστω $A$ γραμμική απεικόνιση με αρνητικές ιδιοτιμές και $\gamma > 0$. Τότε 
    οι ακόλουθες εκφράσεις είναι ισοδύναμες.
    \begin{itemize}
        \item $\norm T < \gamma$ 
        \item υπάρχει λύση $\mathcal{K} = \mathcal{K}^{\intercal}$ στην γραμμική
         ανισότητα πινάκων (\tl{LMI})
        \begin{equation}
            \begin{aligned}
                \mathcal{K} &> 0 \\
                \begin{pmatrix}
                    A^{\intercal} \mathcal{K} + \mathcal{K} A & \mathcal{K} B 
                    & C^{\intercal} \\
                    B^{\intercal} \mathcal{K} & -\gamma I & D^{\intercal} \\
                    C & D & -\gamma I
                \end{pmatrix}
                &< 0. 
            \end{aligned}
            \label{ctl:LMI}
        \end{equation}     
    \end{itemize}
    \label{ctl:lemma}
\end{namedthrm}
\begin{proof}
    βλέπε \cite{Scherer}
\end{proof}

\subsection{Σύνθεση Ελεγκτή}

Στόχος μας είναι η εξασθένιση της επίδρασης των διαταραχών στην έξοδο, δηλαδή να
επιτευχθεί $\norm T < \gamma$. Ορίζοντας κατάλληλα τον ελεγκτικό όρο $u_1$, 
μπορούμε να οδηγηθούμε στο σύστημα κλειστού βρόχου της μορφής (\ref{ctl:linsys})
και να εφαρμοστεί το \ref{ctl:lemma}.

Έτσι, επιλέγεται ένας δυναμικός γραμμικός ελεγκτής ο οποίος παρουσιάζεται στο 
διάγραμμα βαθμίδων στο σχήμα \ref{fig:blk_dgm}.
\begin{figure}[H]
    \begin{center}
        \begin{tikzpicture}[arr/.style={-latex,thin}, line/.style={thick},
        nonterminal/.style={rectangle, minimum size=12mm, thick, 
        draw=black, top color=white, bottom color=white}, >= triangle 90, 
        font = \large]
        
        \matrix[row sep=8mm,column sep=5mm] {
        \node (one) [nonterminal] {$\begin{matrix} 
            \dot{x}_1 =  A_1x_1 +Ew +B_1u_1\\
            z = Hx_1 +Fw + Gu_1\\
            y_1 = C_1x_1 + D_1w
            \end{matrix}$}; \\
        \node (two) [nonterminal] {$\begin{matrix} 
            \dot{x}_2 =  A_2x_2+B_2u_2\\
            y_2 = C_2x_2 + D_2 u_2
            \end{matrix}$}; & \\};

        \draw [-latex] (one.-10)  -| ++ (1,0)   node[pos=1, right, yshift = -29]
         {$y_1$} -- ++ (0,-1.8) -- (two.0);
        \draw [-latex] (two.180)   -- ++ (-1.5,-0) node[pos=1, left, yshift = 
        26] {$y_2$}  |- (one.190);
        \draw [-latex] (one.10) -- ++ (1.1,0) node[pos=0.4, above, yshift = 3] 
        {$z$};    
        \draw [-latex] ++ (-3.65,1.433)  --(one.170) node[pos=0.55, above, 
        yshift = 3] {$w$} ;    
        \end{tikzpicture}
    \end{center} 
    \caption{Διάγραμμα Βαθμίδων}
    \label{fig:blk_dgm} 
\end{figure}

Οι διαστάσεις του συστήματος είναι οι εξής
\begin{alignat*}{6}
    A_1 &\in \euclr{n\times n} \quad E &&\in \euclr{n\times k} \quad B_1 &&&\in 
    \euclr{n\times m}\\
    H &\in \euclr{p\times n} \quad F &&\in \euclr{p\times k} \quad G &&&\in 
    \euclr{p\times m}\\
    C_1 &\in \euclr{q\times n} \quad D_1 &&\in \euclr{q\times k} &&&\\
    &\text{και}\\
    A_2 &\in \euclr{r\times r} \,\,\quad B_2 &&\in \euclr{r \times q}\\
    C_2 &\in \euclr{m\times r} \quad D_2 &&\in \euclr{m \times q}.
\end{alignat*}
% και
% \begin{alignat*}{6}
%     A_1 &\in \euclr{n\times n} \quad E &&\in \euclr{n\times k} \quad B_1 &&&\in 
%     \euclr{n\times m}\\
%     H &\in \euclr{p\times n} \quad F &&\in \euclr{p\times k} \quad G &&&\in 
%     \euclr{p\times m}\\
%     C_1 &\in \euclr{q\times n} \quad D_2 &&\in \euclr{m\times q} 
% \end{alignat*}

Επιδιώκεται, τώρα, ο σχηματισμός του γραμμικού συστήματος κλειστού βρόγχου. Για
$y_2 = u_1$ και $y_1 = u_2$ προκύπτει
\begin{align*}
    &
    \begin{cases}
        \dot{x_1} = A_1 x_1 + E w + B_1 C_2 x_2 + B_1 D_2 u_2\\
        \dot{x_2} = A_2 x_2 + B_2 C_1 x_1 + B_2 D_1 w 
    \end{cases}\Rightarrow\\
% \end{equation*}
% \begin{equation*}
    &
    \begin{cases}
        \dot{x_1} = A_1 x_1 + E w + B_1 C_2 x_2 + B_1 D_2 C_1 x_1 + B_1 D_2 D_1 
        w\\
        \dot{x_2} = A_2 x_2 + B_2 C_1 x_1 + B_2 D_1 w
    \end{cases}    
\end{align*}
το οποίο γράφεται σε μητρωική μορφή ως
\begin{equation*}
    \begin{pmatrix}
        \dot{x_1} \\
        \dot{x_2}
    \end{pmatrix}=
    \begin{pmatrix}
        A_1 + B_1 D_2 C_1 & B_1 C_2 \\
        B_2 C_1 & A_2 
    \end{pmatrix}
    \begin{pmatrix}
        x_1 \\
        x_2
    \end{pmatrix}
    +
    \begin{pmatrix}
        E + B_1 D_2 D_1 \\
        B_2 D_1
    \end{pmatrix} w.
\end{equation*}
Για την έξοδο του συστήματος έχουμε
\begin{align*}
    z =& H x_1 + Fw + G u_1 = H x_1 + Fw + G C_2 x_2 + G D_2 u_2 \Rightarrow \\
    z =& H x_1 + Fw + G C_2 x_2 + G D_2 C_1 x_1 + G D_2 D_1 w 
\end{align*}
το οποίο γράφεται σε μητρωική μορφή
\begin{equation*}
    z =
    \begin{pmatrix}
        H+G D_2 C_1 & G C_2 \\
    \end{pmatrix}
    \begin{pmatrix}
        x_1 \\
        x_2
    \end{pmatrix}
    +
    \begin{pmatrix}
        F + G D_2 D_1 \\
    \end{pmatrix} w.
\end{equation*}
Καταλήγουμε στην τελική μορφή του συστήματος κλειστού βρόγχου
\begin{align*}
    \dot{x} =& Ax + Bw \\
    z =& Cx + Dw.
\end{align*}

Για το σύστημα κλειστού βρόχου ισχύει
\begin{equation*}
    \begin{pmatrix}
        \begin{array}{@{}c|c@{}}
            A & 
            B \\
            \cmidrule[0.4pt]{1-2}
            C & 
            D \\
        \end{array} 
    \end{pmatrix} = 
    \begin{pmatrix}
        \begin{array}{@{}c|c@{}}
            \begin{matrix}
                A_1 + B_1 D_2 C_1 & B_1 C_2\\
                B_2 C_1           & A_2
            \end{matrix} & 
            \begin{matrix}
                E + B_1 D_2 D_1\\
                B_2 D_1
            \end{matrix} \\
            \cmidrule[0.4pt]{1-2}
            \begin{matrix}
                H + G D_2 C_1 & G C_2
            \end{matrix} & 
            \begin{matrix}
                F + G D_2 D_1
            \end{matrix} \\
        \end{array} 
    \end{pmatrix}.
\end{equation*}

Για να επιτευχθεί η συνθήκη $\norm T< \gamma$ ή $\norm{y} < \gamma \norm{w}$, 
από το λήμμα (\ref{ctl:lemma}) θα πρέπει για το κλειστό σύστημα $A, B, C, D$ να 
ικανοποιούνται οι \tl{LMI} (\ref{ctl:LMI}). οι αναζητούμενοι παράμετροι του 
ελεγκτή $A_2, B_2, C_2, D_2$ υπολογίζονται έτσι ώστε να ικανοποιούνται οι 
\tl{LMI}.

Προφανώς ο πίνακας $A$ εξαρτάται από τις παραμέτρους του ελεγκτή. Επιπλέον ο 
πίνακας $\mathcal{K}$ αποτελεί και αυτός μεταβλητή που αναζητούμε. Συνεπάγεται 
από (\ref{ctl:LMI}), ότι ο όρος $A \mathcal{K}$ είναι μη γραμμικός ως προς τις 
ζητούμενες μεταβλητές, καθιστώντας την ανισότητα μη-γραμμική (διγραμμική).

Τα υπολογιστικά εργαλεία που επιλύουν γραμμικές ανισότητες πινάκων, όπως 
υπονοεί και η ονομασία τους, απαιτούν γραμμική μορφή ως προς τις ζητούμενες 
μεταβλητές. Απαιτείται, λοιπόν, μετασχηματισμός των μεταβλητών με τέτοιο τρόπο, ώστε οι 
μετασχηματισμένες μεταβλητές να εισάγονται γραμμικά. Συγκεκριμένα εισάγεται η 
μη-γραμμική απεικόνιση 
\begin{equation*}
    (\mathcal{K}, A_2, B_2, C_2, D_2) \mapsto v = (X, Y, K, L, M, N)
\end{equation*}
η οποία μετατρέπει τις μεταβλητές σε μία εξάδα νέων μεταβλητών $v$ και ορίζονται 
οι μητρωικές συναρτήσεις 
\begin{equation*}
    \begin{pmatrix}
        \begin{array}{@{}c|c@{}}
            \mathbf{A}(v) & 
            \mathbf{B}(v) \\
            \cmidrule[0.4pt]{1-2}
            \mathbf{C}(v) & 
            \mathbf{D}(v) \\
        \end{array} 
    \end{pmatrix} = 
    \begin{pmatrix}
        \begin{array}{@{}c|c@{}}
            \begin{matrix}
                A_1 Y + B_1 M   & A_1 + B_1 N C_1\\
                K               & A_1 X + L C_1
            \end{matrix} & 
            \begin{matrix}
                E + B_1 N D_1\\
                X E + L D_1
            \end{matrix} \\
            \cmidrule[0.4pt]{1-2}
            \begin{matrix}
                H Y + G M & H + G N C_1
            \end{matrix} & 
            \begin{matrix}
                F + G N D_1
            \end{matrix} \\
        \end{array} 
    \end{pmatrix}
\end{equation*}
\begin{equation*}
    \mathbf{X}(v) = 
    \begin{pmatrix}
        Y & I \\
        I & X
    \end{pmatrix}.
\end{equation*}

Το σύστημα γραμμικών ανισοτήτων πινάκων που προκύπτει είναι ισοδύναμο με την 
(\ref{ctl:LMI}) και γράφεται
\begin{equation}
    \begin{aligned}
    \mathbf{X}(v) > 0 \\
    \\
    \begin{pmatrix}
        \mathbf{A}^{\intercal} + \mathbf{A} & \mathbf{B} & 
        \mathbf{C}^{\intercal}\\
        \mathbf{B}^{\intercal} & -\gamma I & \mathbf{D}^{\intercal}\\
        \mathbf{C} & \mathbf{D} & -\gamma I
    \end{pmatrix}
    < 0.
    \end{aligned}
    \label{ctl:LMI2}
\end{equation}


\begin{namedthrm}{Θεώρημα}
    Υπάρχουν παράμετροι ελεγκτή $A_2, B_2, C_2, D_2$ και ένας πίνακας 
    $\mathcal{K}$ που ικανοποιούν την (\ref{ctl:LMI}), αν και μόνο αν υπάρχει 
    $v$ η οποία αποτελεί λύση της (\ref{ctl:LMI2}). Αν $v$ ικανοποιεί την 
    ανισότητα, τότε $I - XY$ είναι μη ιδιάζων πίνακας και υπάρχουν τετραγωνικοί 
    μη ιδιάζοντες πίνακες $U, V$ που ικανοποιούν $I - XY = UV^{\intercal}$. Για 
    κάθε $U, V$
    \begin{align*}
        \mathcal{K} &= 
        \begin{pmatrix}
            Y & V \\
            I & 0
        \end{pmatrix}^{-1}
        \begin{pmatrix}
            I & 0 \\
            X & Y
        \end{pmatrix} \\
        \begin{pmatrix}
            A_2 & B_2 \\
            C_2 & D_2
        \end{pmatrix} &=
        \begin{pmatrix}
            U & X B_1 \\
            0 & I
        \end{pmatrix}^{-1}
        \begin{pmatrix}
            K - X A_1 Y & L \\
            M & N
        \end{pmatrix}
        \begin{pmatrix}
            V^{\intercal} & 0 \\
            C_1 Y & I
        \end{pmatrix}^{-1} 
    \end{align*}
    ικανοποιούν την (\ref{ctl:LMI2}) και γίνεται η εύρεση των ζητούμενων 
    παραμέτρων.
\end{namedthrm}
\begin{proof}
    βλέπε \cite{Scherer}.
\end{proof}
Όλη η παραπάνω υπολογιστική διαδικασία περιγράφεται αναλυτικότερα στο 
\cite{Scherer}.