
\chapter*{Περίληψη}
\noindent Το αντικείμενο της παρούσας διπλωματικής εργασίας είναι ο σχεδιασμός 
ενός συστήματος ελέγχου, για την παρακολούθηση τροχιάς αναφοράς ύψους και 
προσανατολισμού ενός τρικόπτερου αεροχήματος κεκλιμένου στροφείου σε λειτουργία 
αιώρησης.

Το πρώτο βήμα για την επίτευξη του παραπάνω στόχου αποτέλεσε η μοντελοποίηση 
των εξωτερικών δυνάμεων των ελικών που δρούν στο αερόχημα, ενώ καταστρώθηκαν και
οι κινηματικές εξισώσεις που το περιγράφουν. Επιπλέον, έγινε μία βασική 
μοντελοποίηση των κινητήρων με τους οποίους είναι εξοπλισμένο το 
τρικόπτερο. Στην συνέχεια, σχεδιάστηκε ένας αρχικός ελεγκτής, ο οποίος 
βασίζεται στην γραμμικοποίηση με ανάδραση, για την επίτευξη  των επιθυμητών 
αποκλίσεων των μεταβλητές εξόδου με τις τροχιές αναφοράς. Αφού παρουσιάζονται 
τα αποτελέσματα με την συγκεκριμένη μέθοδο, μοντελοποιούνται δύο είδη 
διαταραχών. Έτσι, συγκρίνεται η επίδοση του ελεγκτή με την παρουσία και μη, των 
διαταραχών. Για την εξασθένιση των διαταραχών, συντίθεται ένας $\gamma$-
υποβέλτιστος ελεγκτής βασισμένος στο \tl{Bounded Real Lemma}. Με αυτήν την 
διαδικασία ελαχιστοποιήθηκαν τα μόνιμα σφάλματα, ενώ ενισχύθηκε η ευρωστία του 
ελεγκτή. 

Στα αποτελέσματα, μετά απο μία σειρά προσομοιώσεων, επιβεβαιώνεται η επίδοση του 
ελεγκτή και η επιτυχία του ως προς την αντιμετώπιση των διαταραχών. Επιπλέον, 
παρουσιάζεται και η πειραματική διάταξη, που κατασκευάστηκε για την εκτέλεση 
αρχικών πειραμάτων, τα οποία, δυστυχώς, συμπίπτουν χρονικά με την εκπόνηση της 
διπλωματικής εργασίας και δεν παρουσιάζονται.

\cleardoublepage


\chapter*{\tl{Abstract}}
\tl{The subject of the present diploma thesis is the design of a control system,
for the trajectory tracking of the altitude and the orientation of a tilt-rotor 
tricopter at the state of hovering.}

\tl{The first step towards the above goal was the modelling of all external 
forces of the propellers applied on the tricopter, followed by the derivation 
of the kinematic equations that describe the vehicle's motion. Furthemore,
the dynamic behavior of the motors with which the tricopter is equiped, was also
characterized. Then, an initial control system was developped,
which is based on the feedback linearization technique, to achieve the desired 
error dynamics of the output states. After the presentation of the results of 
the specific methodology, two types of disturbances are modelled and their 
impact is presented, in comparison with the previous results. For the disturbance
cancellation, the initial control is expanded with the construction of a 
$\gamma$-suboptimal controller based on the Bounded Real Lemma. With the 
complete controller, minimization of the trajectory tracking error and increase 
of the robustness of the controller is achieved}.

\tl{On the results chapter, after a series of simulations, the theoretical 
performance of the controller and its success on cancelling disturbances is 
verified. Moreover, the experimental platform is presented,  that was 
construct for future implementations of the developed control system.}

\cleardoublepage


\chapter*{Ευχαριστίες}

Αρχικά, θα θέλαμε να ευχαριστήσουμε τον Καθηγητή κ. Παναγιώτη Σεφερλή, που 
ενθαρρύνοντας μας να ασχοληθούμε με το συγκεκριμένο αντικείμενο, μας έδωσε την 
ευκαιρία να έρθουμε σε επαφή με ένα τόσο ενδιαφέρον θέμα. Επίσης, ιδιαίτερη 
ευγνωμοσύνη οφείλουμε στον Υποψήφιο Διδάκτορα κ. Κώστα Γερμακόπουλο, από τον 
οποίο δεχθήκαμε σημαντική υποστήριξη σε ό,τι αφορά την θεωρία καθώς και την 
υλοποίηση της. Ακόμη, θα θέλαμε να ευχαριστήσουμε τον Υποψήφιο Διδάκτορα κ. 
Παύλο Καπαρό-Τσάφο, από την ερευνητική ομάδα Μη-Επανδρωμένων Αεροχημάτων 
(\tl{UAV-iRC}), για την συνεχή προθυμία του για βοήθεια. Τέλος, οι αληθινές 
ευχαριστίες οφείλονται στις οικογένειές μας, που υπήρξαν στήριγμα σε όλη την 
φοιτητική μας πορεία. 