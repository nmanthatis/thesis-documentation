\chapter{Συμπεράσματα \& Μελλοντικές Προεκτάσεις}
Στην παρούσα διπλωματική εργασία αναπτύχθηκε ένας μη-γραμμικός ελεγκτής για την 
ακολούθηση τροχιών αναφοράς όσον αφορά την αιώρηση ενός τρικοπτέρου. Αρχικά, 
εφαρμόσθηκε γραμμικοποίηση με ανάδραση στο σύστημα για την απόρριψη της 
μη-γραμμικότητας βάσει ενός προτύπου των δυναμικών του αεροχήματος. Στη συνέχεια 
δημιουργήθηκαν τα δυναμικά απόκλισης (\tl{error dynamics}) και το σύστημα πήρε 
γραμμική μορφή. Σε αυτό το σημείο έγινε η στάθμιση του ελεγκτή με γνώμονα τις 
σταθερές χρόνου του συστήματος και τα διαθέσιμα όρια των ενεργοποιητών. Επί 
αυτού, για την απόρριψη των διαταραχών, χρησιμοποιήθηκε το \tl{Bounded Real 
Lemma} για την κατασκευή ενός ελεγκτή ο οποίος φράζει την επίδραση των 
διαταραχών στην έξοδο του συστήματος.  

Εισήχθησαν διαταραχές στο σύστημα όσον αφορά τα κατασκευαστικά χαρακτηριστικά 
του αεροχήματος καθώς και τον τρόπο με τον οποίο παράγονται τα φορτία των ελίκων.
Στη συνέχεια,  έγιναν προσομοιώσεις με διάφορα ρεαλιστικά σενάρια πτήσης του 
τρικοπτέρου. Σύμφωνα με τα αποτελέσματα, ο τελικός ελεγκτής συμπεριφέρεται 
ικανοποιητικά για τις απαιτήσεις της εφαρμογής. Η στάθμιση του ελεγκτή παράγει
τα επιθυμητά αποτελέσματα διατηρώντας σε ανεκτά επίπεδα τις δράσεις των 
ενεργοποιητών.

Σχετικά με την εφαρμογή στον πραγματικό κόσμο, η πλατφόρμα που κατασκευάστηκε 
είναι πλήρως λειτουργική. Ο αλγόριθμος για τον ελεγκτή πτήσης \tl{Pixhawk Cube} 
έχει ήδη αναπτυχθεί σε γλώσσα προγραμματισμού \tl{C}. Δυστυχώς, η χρονική 
περίοδος εκπόνησης αυτής της εργασίας συμπίπτει με τα πρώτα δοκιμαστικά για το 
τρικόπτερο. Μέχρι στιγμής έχουν γίνει κάποιες πρώτες ανεπιτυχείς προσπάθειες 
πτήσης καθώς παρατηρούνται διάφορα σφάλματα που αναγκαστικά δημιουργούνται κατά 
τη μετάβαση από τη θεωρία στην πράξη.

Για το άμεσα κοντινό μέλλον, προτεραιότητα έχει να πραγματοποιηθεί μια πρώτη 
επιτυχής αιώρηση του τρικοπτέρου. Τότε, σε δεύτερο χρόνο θα μπορέσουμε να 
αξιολογήσουμε την επίδοση του ελεγκτή και να σταθμίσουμε τις παραμέτρους του. 
Εάν κριθεί αναγκαίο να προστεθεί κάποια ιδιότητα στον ελεγκτή μας η ακόμη να 
αντικατασταθεί.

Για τη περαιτέρω ανάπτυξη του ελεγκτή πτήσης θα πρέπει να συμπεριληφθεί ο 
έλεγχος όλων των μεταφορικών συντεταγμένων του αεροχήματος. Μπορούν, επίσης, να 
εισαχθούν μοντέλα διαταραχών για τις μετρήσεις των αισθητήρων καθώς και 
των αεροδυναμικών δυνάμεων που επιδρούν κατά την αιώρηση. Μία ακόμη διαταραχή 
αποτελεί η μοντελοποίηση του τρικοπτέρου ως στερεό σώμα, έτσι μπορεί κατά την 
πτήση να ενεργοποιούνται μη-μοντελοποιημένα δυναμικά κατά την λειτουργία του 
κεκλιμένου στροφείου. Έπειτα, για την ικανοποίηση των απαιτήσεων λειτουργίας του
\tl{MPU}, θα πρέπει να αναπτυχθεί μια λειτουργία για την μετάβαση από την 
κατάσταση αιώρησης στην πτήση ώς αερόχημα σταθερής πτέρυγας καθώς και την 
σύνθεση ενός ελεγκτή πτήσης για την κατάσταση σταθερής πτέρυγας του αεροχήματος.